\documentclass[10pt,twocolumn]{article} 

% use the oxycomps style file
\usepackage{oxycomps}

% use packages for timeline
\usepackage{array, booktabs, longtable}
\usepackage{graphicx}
\usepackage[x11names, table]{xcolor}
\usepackage{caption}
\DeclareCaptionFont{blue}{\color{LightSteelBlue3}}

\newcommand{\foo}{\color{LightSteelBlue3}\makebox[0pt]{\tiny\textbullet}\hskip-0.5pt\vrule width 1pt\hspace{\labelsep}}
\newcommand{\bfoo}{\raisebox{2.1ex}[0pt]{\makebox[\dimexpr2\tabcolsep]%
{\color{LightSteelBlue3}\tiny\textbullet}}}%
\newcommand{\tfoo}{\makebox[\dimexpr2\tabcolsep]%
{\color{LightSteelBlue3}$\boldsymbol \uparrow $}}%

% read references.bib for the bibtex data
\bibliography{references}

% include metadata in the generated pdf file
\pdfinfo{
    /Title (The Occidental Computer Science Comprehensive Project: Goals, Format, and Advice)
    /Author (Justin Li)
}

% set the title and author information
\title{The Occidental College Computer Science Comprehensive Project Proposal: \\ Predicting Cryptocurrency Prices for Stock Trading Using Machine Learning}
\author{Odelia Putterman}
\affiliation{Occidental College}
\email{putterman@oxy.edu}

\begin{document}

\maketitle

\begin{abstract}
    This paper is my Occidental College Computer Science Comprehensive Project (COMPs) Proposal for my project: \textit{Predicting Cryptocurrency Prices for Stock Trading Using Machine Learning}. This project proposal has seven components, one for each of: introduction, problem context, technical background, methods, evaluation, ethical considerations, and software documentation.  I begin by introducing the project, then I address the technical aspects of realizing this work, and I end with the ethical concerns regarding this assignment. In tackling each of these sections, I outline the motivation for this work, ethical considerations, and how to produce this project.
\end{abstract}


\section{Introduction}

For my senior-year Occidental College Computer Science Comprehensive Project (COMPs), I have chosen to build an index fund for cryptocurrencies, with rebalancing occuring dynamically in response to crypto price predictions from a machine learning (ML) algorithm. This proposed index fund will utilize the volatility of cryptocurrencies to maximize gains by dynamically trading index fund stocks rather than rebalancing on a preset schedule.

\section{Problem Context}

Investing is a means by which people can grow their money exponentially, but it also holds the potential for significant loss, especially in the context of day trading.

\subsection{Index Funds: Explained}

As it stands, it's difficult, if not impossible, to predict the exact performance of individual stocks. As such, people invest in index funds, collections of stocks which follow some set of rules for construction, to mirror the whole market rather than a particular holding. Such funds are generally mutually pooled, with the stock holdings shared across many holders.

Money that goes into the fund is invested in all the stocks the fund holds, growing the diversification of the fund and minimizing its dependence on individual stock performance. Further, index funds are periodically re-balanced to offset losses and risk, and companies are sold once they leave these predefined rules or parameters discussed.

\subsection{Cryptocurrencies: Unique Opportunity and Approach}

Index funds are generally constructed using fundamental measures of a company (aka fundamentals), such as cash flow, sales, and dividends. Cryptocurrencies, however, have none of these intrinsic metrics. Since the cryptocurrencies themselves are a currency rather than a materialized product, there are no sales or dividends to be found. Their only measure is price. Accordingly, any approach to creating and maintaining a cryptocurrency index fund must be revolutionary in nature.

Cryptocurrencies are incredibly volatile, making re-balancing especially important. Their instability also offers a unique opportunity for leveraging their turning points to optimize gains if high-accuracy price prediction can be achieved. If we could predict cryptocurrency prices in advance, we could leverage this knowledge to maximize gains while keeping a sense of security with the breadth of the fund.

Typically, index funds are rebalanced on a consistent, preset schedule, such as the third Friday at the end of each calendar quarterly. But, what if we could predict cryptocurrency prices to decide how and when to rebalance such a cryptocurrency fund instead? If this succeeded, the results would be revolutionary in nature and a first-of-its-kind for maintaining a balanced, high-return cryptocurrency fund.

People have previously attempted to predict the behavior of cryptocurrencies with some success. I aim to follow in their steps with a slightly different approach and following implementations. Such prior work is discussed below under \textbf{Prior Work}.


\section{Technical Background}

Here I introduce the technical background required to fully comprehend the methods this project undertakes.

\subsection{What Are Cryptocurrencies?}

This comps project involves cryptocurrencies, a relatively new phenomena stemming from advances and uses of blockchain technologies. Cryptocurrencies were first created for their supposed safety and security arising from their basis in blockchain technology. According to \citetitle{WhatIsBlockchain}, ``Blockchain is a shared, immutable ledger that facilitates the process of recording transactions and tracking assets in a business network". Each transaction is recorded as a ``block" of data, connected to the blocks before and after it in an irreversible chain (aka a blockchain).

\subsection{Sentiment Data and Sentiment Analysis}

Sentiment data is essentially the same as it sounds: it is data which signifies sentiment. Sentiment analysis, similarly, is the use of computational algorithms to study subjective information (i.e. sentiment). It is particularly important for understanding the social sentiment of a given thing \cite{SentimentAnalysisConcept}. In the context of this project, the sentiment data is data which can be analyzed to search for crowd feelings on cryptocurrencies, while the sentiment analysis is the analysis of this data to learn attitudes towards cryptocurrencies.

\subsection{What are LSTMs?}

Long-short term neural networks (LSTMs) are a type of neural network or, more specifically, recurrent neural network (RNN). RNNs are a type of neural network capable of learning order dependence, which is key in sequence prediction problems. They differ from traditional feedforward neural networks, or multilayer perceptrons (MLPs), in their inclusion of feedback connections. To be an RNN, a system must

\begin{enumerate}
    \item be able to store information for an arbitrary period;
    \item be resistant to noise (i.e. random inputs, outliers, etc.); and
    \item have trainable system parameters.
\end{enumerate}

In RNNs, the \emph{context} is learned, meaning an RNN contains cycles which feed the network outputs, or activations, from a prior time step as input to the network for future time steps. However, they fail to ``remember" things from far back. RNNs are susceptible to backpropagated error. RNNs have hidden layers which can decay or blow up when circling around a feedback connection \cite{GentleIntroductionToLSTMNetworks}. Essentially, layers in RNNs with small enough gradient updates stop learning, usually in relatively early-on layers, causing short-term memory, or what is known as the vanishing gradient problem, while layers with large enough gradients exponentially grow towards infinity, causing a blow-up problem and an unrealistic expectation for reliance on unimportant, or not as important as is implied, information \cite{IllustratedGuideToLSTMs}.

LSTMs are a subset of recurrent neural networks (RNNs), similar in most respects, introduced in 1997 by Hochreiter and Shmidhuber \cite{UnderstandingLSTMs}. LSTMs differ from RNNs in their ability to solve the vanishing gradients and exploding gradients problems. Like RNNs, they learn order dependence in sequence prediction problems, only they are far more effective at retaining both long-term and short-term memory. They can keep information about the past for a unfixed period, to be decided by the input data weights as feedback connections are formed throughout the neural network operations; they use contextual information flexibly. LSTMs can traverse longer time steps to store important information with a constant error flow, avoiding the RNN problem or exponential blowing-up and decay \cite{GentleIntroductionToLSTMNetworks}. These feedback connections, or loops, allow for the persistence of information in traditional RNNs, and, correspondingly, LSTMs. RNNs, and, accordingly, LSTMs, can be thought of as multiple copies of the same network, each passing information to the succeeding network. When the gap in knowledge from the prior input information to the next is relatively small, an RNN may succeed. But substantial research shows LSTMs far outperform typical RNNs when more context is needed \cite{UnderstandingLSTMs}.

\subsection{Index Fund Rebalancing}

Index fund rebalancing is the process of rebalancing is the process of selling and buying stocks to balance the distribution across a fund.

There are many ways to rebalance an index fund. The usual choice is \textbf{calendar rebalancing}. In calendar rebalancing, the index fund holdings are reviewed on a preset calendar basis and adjusted to their original form. The actual rebalancing is done by examining issues such as transaction costs and drifts.

Another type of rebalancing, later discussed, is \textbf{percentage-of-portfolio rebalancing}. According to \citetitle{TypesOfRebalancingStrategies}, percentage-of-portfolio rebalancing is a ``rebalancing schedule focused on the allowable percentage composition of an asset in a portfolio. Every asset class, or individual security, is given a target weight and a corresponding tolerance range."

There are many other types of rebalancing algorithms, but these are the two most important rebalancing algorithms to understand for the scope of this project.

\section{Methods}

I propose to predict cryptocurrency prices using machine learning, specifically an LSTM neural network, and sentiment analysis. Because, as previously mentioned, cryptocurrencies don’t have fundamentals, we must evaluate and try to predict their behavior using other metrics for input data. Here, I propose the use of sentiment analysis.

\subsection{Language of Choice}

My language of choice for this project is Python using Google Colab. Python is my language of choice for its abundant scraping and machine learning libraries/tools in addition to its simplicity of use. Accordingly, all libraries discussed are Python libraries. Google Colab will also be used for its access to GPUs for network training purposes.

\subsection{Sentiment Analysis Input Data}

Originally I thought to use both sentiment analysis and current events data for cryptocurrency price prediction. But, after realizing the complexity involved in this project, I chose to stick with just sentiment data since, ideally, sentiment should mirror current events; any major disarray which may affect crypto prices will likely be expressed in sentiment towards the currency.

Such sentiment, in this case, should convey feelings towards or about cryptocurrencies. This sentiment can be sourced form news articles, social media posts, and other messaging. Input data for sentiment analysis should include, but is not limited to:

\begin{itemize}
    \item The number of positive and/or negative news articles about specific cryptocurrencies/cryptocurrency (as a concept) generally;
    \item The number of positive and/or negative social media posts about specific cryptocurrencies/cryptocurrency (as a concept) generally; and
    \item The number of positive and/or negative crypto blog postings about specific cryptocurrencies/cryptocurrency (as a concept) generally.
\end{itemize}

As I continue my project research, I plan to add more input sources on which to perform sentiment analysis to gather crowd feelings towards cryptocurrencies.

\subsection{Obtaining Data for Sentiment Analysis}

Now that we have determined which data to analyze, we must consider how this data will be acquired. Ideally, there would be preset datasets with this information. But, I have yet to find any with the breadth, depth, and upkeep necessary for such a machine learning algorithm to succeed. So, unless I find an ideal dataset, scraping is my preferred method for training and maintenance data acquisition. I don’t want my project to be about how to scrape data, rather I’d like to make this as painless as possible. To prevent scraping itself becoming a massive project, I plan to use existing libraries for help tackling this problem, particularly PyGoogleNews, but my backup libraries are:

\begin{itemize}
    \item NewsCatcher;
    \item FeedParser; and
    \item NewsPaper3k.
\end{itemize}

Once I have this scraped data, I plan to convert it to a csv file with columns for date and text.

Cryptocurrencies are a global phenomenon, so we should not limit our scope of search to US-based sites. But, this is still a college project with limitations to its scope, so I plan to start with sourcing articles written in English from mostly US-based cites.

I plan to narrow my list of sentiment data sources and scrape from these to get the training data and periodically retrieve classification data for price prediction. Below, I list sources I suggest be scraped for input data to the sentiment analysis.

\subsubsection{News Sources}
\label{sec:newssources}

\begin{enumerate}
    \item The New York Times;
    \item The Economist; and
    \item CNN.
\end{enumerate}

\subsubsection{Social Media}

\begin{enumerate}
    \item Instagram;
    \item Facebook; and
    \item Twitter.
\end{enumerate}

\subsubsection{Crypto Blogs}

\begin{enumerate}
    \item Cointelegraph;
    \item NewsBTC; and
    \item CryptoNinjas.
\end{enumerate}

I am considering scraping solely the headlines for these news articles and blog posts to simplify the process.

Most of these should be available to be openly viewed or have APIs which could be connected for scraping. This scraping will source text containing key words, such as "crypto", "cryptocurrency", or the names of any of the cryptocurrencies to be added to this fund. This scraping would look for text with occurrences of related words for each of the aforementioned sources and add them to this csv under the given date. Data with only broad words such as "cryptocurrency" would be concatenated and added under a column named "cryptocurrency", while text referencing specific-crypto coins would be stored under a column of that cryptocurrency's name. This csv file would be used as input for Sentiment Analysis.

\subsection{Sentiment Analysis to LSTM Input Data}

To decipher this sentiment, I will use an existing natural language processing (NLP) algorithm. My decided library for this sentiment analysis is TextBlob, but I am open to using other libraries, such as NLTK (Natural Language Toolkit), SpaCy, Stanford CoreNLP, and Genism, to compare their outputs to find the best performing model.

This sentiment analysis will append the previous csv file that lists the date and related text for each cryptocurrency and cryptocurrency broadly, adding in new columns for subjectivity, negative sentiment, positive sentiment, and neutral sentiment, which will then be passed forward to the LSTM neural network for cryptocurrency price prediction and training.

\subsection{Predict Cryptocurrency Prices with LSTM Neural Network}

Deep learning models are novel in their classification capabilities, and with decreasing compute times, they are an increasingly affordable option. But traditional deep learning models are not able to store contextual information, lacking the persistence necessary for sequential predictions, such as stock price predictions. The need for a method capable of time-sensitive sequential predictions leads us to recurrent neural networks (RNNs), or, more fittingly, long-short term memory (LSTM) neural networks.

LSTMs are a great candidate for time-series predictions such as stock price predictions. Since they are able to use historical information and continuously contextually decide what information is relevant, they are able to more effectively return on such sequential problems. To solve the stock price prediction problem.

For all these reasons, to predict cryptocurrency prices, I suggest the use of LSTM neural networks. Possible libraries for building and training an LSTM neural network to perform this task are:

\begin{itemize}
    \item TensorFlow;
    \item PyTorch;
    \item NeuroLab; and
    \item Scikit-Neural Network.
\end{itemize}

I want to use TensorFlow specifically since it is very compatible with Google Colab and Cuda, allowing me to leverage the GPUs available there.

The LSTM will be inputted the output from the sentiment analysis step (new sentiment columns) and add to the input csv file future dates and corresponding cryptocurrency price predictions by cryptocurrency, to be inputted to the rebalancing trading algorithm.

\subsection{Feed ML Outputs to Trading Algorithm}

A key idea to the success of this project's proposed index fund is the use of cryptocurrency price predictions to rebalance a proposed index fund dynamically rather than on a preset schedule. My choice of rebalancing is a mixture between calendar rebalancing and percentage-of-portfolio rebalancing, where basically every given period of time (ideally deciphered by an ML algorithm as when prices are dramatically changing), I would rebalance the portfolio to mitigate risk but maximize profits with percentage-of-portfolio rebalancing. This will result in the proposed cryptocurrency index fund.

\section{Evaluation}

This project has three main components: sentiment analysis, long-short term neural network cryptocurrency price predictions, and index fund rebalancing. Accordingly, I will be evaluating my project at each of these steps.

To evaluate the sentiment analysis component, I will create an evaluation csv set which I cross-check myself to run through this sentiment analysis step and compare my sentiment/subjectivity score against the algorithm-generated one. I may, potentially, do this for outputs from a few libraries to decipher which best predicts sentiment for my uses. Moreover, I plan to create a trained model for sentiment deciphered from different algorithms to see which returns the best results.

To evaluate the success of these models, I plan to feed the model data from after the training period and compare the outputted cryptocurrency price prediction results against the actual cryptocurrency prices from that time to see if the LSTM neural network succeeded.

To evaluate the success of the cryptocurrency index fund itself, I will compare the year-to-date (YTD) success of the sentiment analysis based LSTM neural network and index fund trading algorithm - potentially multiple funds based on different NLP algorithms, ML algorithms, and training datasets - with the YTD success of common cryptocurrency index funds (i.e. ProShares Bitcoin Strategy ETF, Grayscale Bitcoin Trust, Bitwise 10 Crypto Index Fund...).

Based on how each model performed, we can answer some follow up questions to maximize the accuracy of the cryptocurrency price prediction algorithms and index fund:

\begin{itemize}
    \item How did the price-prediction-based trading algorithm perform?
    \item Should we use one or another NLP library for sentiment analysis?
    \item Should we use one or another cryptocurrency price prediction solely to inform the cryptocurrency index fund trading algorithm?
\end{itemize}

\section{Prior Work}

LSTMs are one of the most widely employed methods for sequential data problems and time-series forecasting. In this section, we will focus on prior work in using LSTMs for stock price prediction.

\subsection{Sentiment Analysis}

According to efficient market hypothesis, stock prices are driven by news rather than past and present prices; they change based on news of changing fundamentals and human psychology (investor reactions to this news) \cite{LSTMSentimentAnalysis}. This suggests our input data to an LSTM for stock price predictions should include fundamentals and sentiment analysis so we can model prices based on changes in news and sentiment which gauges crowd feelings towards a given stock or industry.

\cite{ReviewAndTaxonomyOfPredictionTechniques} recently employed fundamental analysis for stock price prediction, analyzing company profiles, industry, politics, the economy, and social media according to \cite{LSTMSentimentAnalysis}. However, as \cite{LSTMSentimentAnalysis} mentions, this type of fundamental data is usually rather abstract or unstructured, making it hard to fit into a form LSTMs can use. Instead, \cite{LSTMSentimentAnalysis} suggests using NLP methods to distill relevant information. Sentiment analysis is a strong candidate currently used to analyze text for emotional sentiments and reactions to ``understand" human feeling. From this ``understanding", we can examine how a stock may perform based on human feelings towards this stock. If, for example, negative sentiment about bitcoin is spreading, this may signal to us a coming drop in the bitcoin stock price. As such, sentiment analysis of crowd psychology can monitor public opinion \cite{LSTMSentimentAnalysis}.

\subsection{RNNs and LSTMs}

RNNs, particularly their LSTM derivative, have been shown to outperform traditionally used models, such as support vector machines (SVMs) in accurately predicting for time-series problems \cite{LSTMSentimentAnalysis}. By feeding a combination of fundamental and sentiment analysis to LSTMs to analyze, researchers have made promising strides towards success in predicting stock prices.

Similar to what I propose to do, \citetitle{LSTMSentimentAnalysis} used a long short-term memory neural network with text sentiments and historical stock transactions to predict stock prices. The authors argue in the paper's introduction for the importance of aggressive investing strategies and, hence, the usefulness of accurate stock trend forecasting. They use fundamental analysis and technical analysis to forecast stock prices in addition to an LSTM, which was applied to this forecasting. Using text sentiment from blog articles, social media, online forum posts, and product reviews, the authors aimed to understand online public opinions. They used an existing NLP algorithm (BERT) to classify the texts into sentiment categories by counting the emotional words that appear in a given sentence, then calculating a score for each emotional word, and, with these individual scores, output the condition of the sentence. Further, this article addressing preprocessing of data to get it into a form where it can be passed to the ML algorithm, which is very useful for me to apply to my own product algorithm.

\subsection{Other Models}

\citetitle{ForecastingAndTradingCryptoWithML} examined the predictability of three major crypto contenders: bitcoin, ethereum, and litecoin. They focused on times of extreme upheaval and used linear models, random forests, and support vector machines for their predictive work. Of the 18 models they built, 13 models had a success rate over 50\%, affirming the promise of ML algorithms for cryptocurrency stock price prediction. In their introduction, they explain the proposal of sentiment analysis for stock price prediction, which is what motivated me to choose the same for my project.

\subsection{Trading Algorithms}

There are many possible trading algorithms to use for index fund rebalancing. 

\subsubsection{Mean-Variance Optimization}

One such trading algorithm is mean-variance optimization. \citetitle{MeanVarianceAlgorithmicTrading} implores this algorithm, implementing Markowitz's portfolio optimization (MPT) and the Modern Portfolio Theory are  in python as trading strategies with the \textit{zipline} library. The general goal of MPT is maximize returns while minimizing risk. This method ignores transaction costs, which, for an index fund, do need to be considered and disallows short-selling.

\subsubsection{Constant Rebalanced Portfolios}

Constant Rebalanced Portfolios are another trading algorithm, achieved by \citetitle{ConstantRebalancedPortfolios}. The idea explained in this article is:

Say we have three stocks, and we believe each will perform equally well, so we split our money in three, making our holdings $\{1/3, 1/3, 1/3\}$. Let's say one of these three stocks does significantly better than the rest, making our holdings: $\{1/2, 1/4, 1/4\}$. The idea here is we believe this disproportionate return was just chance, and so we sell off some of this stock and rebalance to our original $\{1/3, 1/3, 1/3\}$.

The use of this trading algorithm has historically allowed for cutting losses and maximizing gains dynamically.


\section{Ethical Considerations}

When considering the ethical considerations of predicting cryptocurrency prices for stock trading using machine learning, we must break the question into sub-tasks. Since this project advocates for the use of cryptocurrencies, we must consider the ethics of cryptocurrencies themselves. Further, we must look into the ethics of stock trading and the investor obligations that come into play when building an index fund, especially one built with machine learning.

\subsection{Nature of Cryptocurrency}

Due to the immutability of blockchain, and, hence, cryptocurrencies, the risk of tampering by bad actors is eliminated, creating a trustworthy network. Traditionally, information storage and retrieval systems were susceptible to cyberattacks. Moreover, without transparency on behalf of the record-management company or system, record tampering would be harder to catch and verify. The open-book nature of blockchain allows for widespread verification of all transactions. Coupled with the blockchain's immutable nature, with no user nor managing power having the ability to alter or delete a transaction, blockchain technologies, and, hence, cryptocurrencies, are extremely safe from a tampering standpoint \cite{WhatIsBlockchain}. This addresses transparency on behalf of the cryptocurrency operations, but it does not cover the ethical considerations with regard to the security of such assets.

A second consideration is the instability of where cryptocurrencies are hosted. One Washington Post Article, \citetitle{TrackingStolenCrypto}, details the commonality of crypto heists. Hackers have previously hacked into cryptocurrency trading networks, getting away with massive sums. In August of 2021, Poly Network, a trading platform for popular cryptocurrencies, was hacked, with the hackers taking 
\$610 million in crypto, which they quickly converted to a ``stable coin". Luckily, the ``stable coin" Tether worked with authorities to release the holdings to the rightful owners. Nonetheless, this heist broke the belief that cryptocurrency is impossible to trace (public ledgers do not retain account holder information) \cite{TrackingStolenCrypto}. The anonymity of cryptocurrencies may be ethical, guarding the identity of its users or not, allowing for bad actors to use crypto for illicit activities, but it is certain that their online holdings make them susceptible to being stolen, further questioning the security of their nature.

Despite the secure theoretical ideas behind cryptocurrencies, their value is often in flux. Ethicists have previously questioned the ethics of cryptocurrencies themselves. Professor Tobey Karen Scharding shared her evaluation of bitcoin from an ethical perspective in an interview at Rutgers Business School:

\begin{quote}
    With bitcoin, I felt there were too many uncertainties... [The] evaluation was pretty negative because bitcoin didn't really have any way to secure its value, and so it wouldn't be able to fulfill the ethical purpose of currency on Fichte's account of stabilizing and securing these exchanges over generations. \cite{IsBitcoinEthical}
\end{quote}

Professor Scharding refers here to the Fichte account, which explains that currency should allow people to live their lives with secure access to basic goods and services and allows them to enjoy life and exchange goods with other people. By Professor Scharding's logic, cryptocurrency falls short of being an ethical currency because its often-changing value means that one day it could secure these assets and more for a given holder while the next day it could be deemed worthless, leaving the holder unable to meet these basic needs. That said, she believes cryptocurrencies could become ethical if an entire nation takes on cryptocurrency and stabilizes its value \cite{IsBitcoinEthical}.

\subsection{Investor Obligations}

Managing an index fund holds a moral obligation to the holders to aim to grow their investment. Cryptocurrency price prediction carries inherit risk. A major concern is the volatility of cryptocurrencies. Though we seek to leverage this volatility, it could also be a major pitfall if prices are incorrectly predicted. In this proposed fund, if the crypto prices are wrongly predicted, funds will likely get misallocated and, in turn, lose the investor money. If the machine learning algorithm wrongly predicts the coming crypto prices, we risk losing investor money and being an unethical project, as it promotes investing with no return. To uphold our ethical obligation to investors, we must strive for the highest level of accuracy in this work.

\subsubsection{Model}

Cryptocurrency prices are extremely volatile, making their price prediction a uniquely difficult challenge. The likelihood that this comps project were to succeed in accurately predicting cryptocurrency prices consistently is incredibly low. This leaves a high likelihood  of inaccurate predictions at some point, if not frequently, which has greater implications on the ethical consideration of investor obligations, model transparency, and security. Promoting the practice of placing a monetary commodity in cryptocurrencies is ethically dubious, as it holds high risks of loosing this money, threatening the financial security of such investors.

To make this project more ethical, there is a need for a lot of transparency on the risks involved in investing in this comps project intended fund due to the accuracy rates of the developed machine learning model and the extremely volatile nature of the assets involved.

\subsection{Investing}

To explore the ethics of this investing-based project, we ask, \textit{is investing itself fair?} In investing, we subscribe to a system where the wealthy have an infinitely easier time to build more wealth than the poor do. The stock market is an inherently unfair playing field. Due to the compound nature of investing and investments, it is infinitely easier to become wealthy if you are already wealthy than it is to turn a little into a lot; the stock market favors those who start off with more. By this logic, it appears to be an unethical system.

We can, however, approach ethical investing from another standpoint. Rather than concern ourselves with the ethics of investing itself, we can ask how to ethically invest money. There is a growing field of learning around \textit{ethical investing}. Ethical investing has many different approaches, but the idea is to use investing as a tool to do good. Manisha Thakor, a financial planner and consultant, says, according to \citetitle{LimitsOfEthicalInvesting}, ``The broad idea behind this style of investing is a belief that you can generate meaningful, measurable, societal outcomes while also generating a healthy profit". You can put your money towards companies addressing an ethical issue, such as climate change, racism, or workplace inequality while also generating wealth. The three overarching areas for ethical investments are environment issues, societal issues, and governance issues \cite{LimitsOfEthicalInvesting}. Cryptocurrencies do not fall under environment issues to the best of my knowledge, but they can, arguably, fall under societal and governance issues.

For starters, cryptos can help in enabling financial inclusion. As \citetitle{CryptocurrenciesFinancialInclusion} says, ``The crypto economy is leading to the development of an alternative financial and technological infrastructure that is global, open source, and accessible to all who have access to the internet, regardless of nationality, ethnicity, race, gender, and socioeconomic class... not enough of us are rolling up our sleeves and getting involved with building this new global and inclusive open financial system". By this framework, cryptocurrencies have the power to do extreme good, and investing in them allows us to "[roll] up our sleeves and [get] involved with building this new global and inclusive open financial system", allowing for greater financial inclusion and opportunity worldwide, a pivotal societal and governance problem.


\subsection{Accessibility}

Another ethical concern this project faces is accessibility. We must consider:

\begin{itemize}
    \item Is it fair that only some have access to crypto price predictions?
    \item Is it ethically right to make such crypto price predictions widely available?
    \item If crypto price predictions are made widely available, will the prediction itself alter the crypto prices, creating self-fulfilling, `announcement-driven' outcomes?
\end{itemize}

We navigate each of these questions in the below sections.

\subsubsection{Power Distribution}

As it stands, ``the rich get richer". If this comps project achieved any level of success in predicting cryptocurrency prices, there would be a question of \textit{who} gets access to these predictions. If these predictions were offered as a subscription service with a high buy-in, the buyers would most likely be those wealthier individuals, giving the elite investors a tool not readily available to the average investor, making it continuously easier for the wealthy to keep get wealthier while the poor get poorer. If these predictions or this fund were expensive and exclusive, we risk contributing to the wealth divide and creating further disparity in power distribution. If instead we choose to make this tool openly accessed, we run into other ethical problems.

\subsection{Self-Fulfillment}

Imagine now that everyone had access to this crypto price prediction tool and that it operated with some meaningful level of accuracy. If these predictions were so easily accessed, it is entirely possible that the predictions would become self-fulfilling cycle where:

\begin{enumerate}
    \item Price predictions are publicized;
    \item People buy stock when its predicted to be rising in price and sell it when it is predicted to fall in the coming future;
    \item These prediction-based purchases drive up or down the crypto.
\end{enumerate}

If this were to happen, no \textit{actual} predicting would be helpful, and all people would not be benefited by such predictions. As such, it is important that the price prediction component of this comps project be kept private, with only the fund performance made available.

\section{Proposed Timeline}

\renewcommand\arraystretch{1.4}\arrayrulecolor{LightSteelBlue3}
\captionsetup{font=blue, labelfont=sc, labelsep=quad}
\begin{longtable}{@{\,}r <{\hskip 2pt} !{\foo} >{\raggedright\arraybackslash}p{5cm}}
 \caption{Timeline} \\[-1.5ex]
\toprule
\addlinespace[1.5ex] 
 \multicolumn{1}{c!{\tfoo}}{}& \\[-2.3ex]
Aug. 1 - Aug. 14 & Set up repository for scraping data from chosen sources on keywords\\
Aug. 15 - Aug. 28 & Collect data and pre-process data into data sets for sentiment analysis \\
Aug. 29 - Sep. 11 & Perform sentiment analysis on pre-processed data set (csv format)\\
Sep. 12 - Sep. 25 & Build out LSTM neural network model\\
Sep. 26 - Oct. 9 & Train LSTM neural network on data set(s)\\
Oct. 10 - 23 & Build index fund rebalancing algorithm\\
Oct. 24 - Nov. 6 & Pass LSTM neural net outputs to rebalancing algorithm\\
Nov. 7 - Nov. 20 & Evaluate index fund\\
Nov. 21 - Dec. 4 & Finish COMPs paper and present findings\\
 \multicolumn{1}{c!{\bfoo}}{}&
\end{longtable}



\printbibliography 

\end{document}